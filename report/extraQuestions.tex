\documentclass[12pt]{article}
% Эта строка — комментарий, она не будет показана в выходном файле
\usepackage{ucs}
\usepackage[utf8x]{inputenc} % Включаем поддержку UTF8
\usepackage[russian]{babel}  % Включаем пакет для поддержки русского языка
\usepackage{amsmath, amsthm, amssymb, amsfonts}
\usepackage{multirow}
\usepackage{pbox}
\usepackage[showframe=false]{geometry}
\usepackage{changepage}
\usepackage{graphicx}
\usepackage{float}
\usepackage{hyperref}
\hypersetup{
	colorlinks,
	citecolor=blue,
	filecolor=blue,
	linkcolor=blue,
	urlcolor=blue
}

\date{}
\author{}

\renewcommand{\thesubsection}{\arabic{subsection}}
\makeatletter
\def\@seccntformat#1{\@ifundefined{#1@cntformat}%
	{\csname the#1\endcsname\quad}%    default
	{\csname #1@cntformat\endcsname}}% enable individual control
\newcommand\section@cntformat{}     % section level 
\makeatother

\begin{document}
	\subsection{Вопросы}
	\begin{enumerate}
		\item Сколько действий требуется для построения интерполяционного полинома в форме Ньютона?
		\item В программе Вам требуется вычислять значения многочлена Лагранжа и кубического сплайна в большом числе точек. Сколько операций требуется для вычисления значений многочлена Лагранжа и кубического сплайна в $n^2$ точках? Как можно ускорить вычисления?
		\item Как построить базисные функции аналогичные $\phi_k$ в глобальной полиномиальной интерполяции?
	\end{enumerate}

	\subsection{Разбор}
	\begin{enumerate}
		\item Полином Ньютона представим в форме 
		$$ P_n(x) = A_0 + A_1(x-x_0) + ...+ A_n(x - x_0)...(x - x_{n-1}) $$
		$$  A_i = \sum_{k=0}^i\frac{f(x_k)}{w_{k,i}} $$
		$$  w_{k,i} = (x_k - x_0)...(x_k - x_{k-1})(x_k - x_{k+1})...(x_k - x_i) $$
		
		Для того чтобы заранее посчитать коэффициенты $w_{k,i}$ нужно провести $n^2$ операций умножения. 
		Это возможно благодаря оптимизации, основывающейся на формуле $w_{k,i+1} = w_{k,i} \cdot (x_k - x_{i+1})$.
		При известных $w_{k,i}$ для подсчета $A_i$ требуется $\frac{n(n - 1)}{2}$ операций.
		Таким образом для построения интерполяционного полинома в форме Ньютона нужно $O(n^2)$ операций.
		
		\item 
		\begin{enumerate}
		\item Рассмотрим \textit{интерполяционный полином Лагранжа}
		$$ L_{n}(x)=\sum_{k=0}^{n} c_{k}(x) y_{k} $$ 
		$$ c_{k}(x)=\prod_{j=0, \atop j \neq k}^{n}\left(\frac{x-x_{j}}{x_{k}-x_{j}}\right) $$
		
		Вычисление $c_k(x)$ занимает $n$ итераций. Вычисление $L_n(x)$ занимает $n^2$ итераций соответственно. Таким образом алгоритмическая сложность вычисления значения полинома в точке --- $O(n^2)$.
		
		В каноническом варианте вычисление значений интерполяционного полинома в $N^2$ точках займет $N^2 \cdot O(n^2)$ где $n$ --- количество узлов сетки, $N^2$ --- количество точек в которых вычисляется значение полинома.
		
		Для ускорения вычислений можно привести многочлен Лагранжа к каноническому виду $$L_n(x) = \sum_{k = 0}^{n} a_k x^k. $$		
		При вычисления значения многочлена Лагранжа приведенного к каноническому виду можно воспользоваться тем, что $x^{k + 1} = x \cdot x^{k}$
		и последовательно вычислять слагаемые, на каждое из которых будет производиться только $2$ операции умножения.
		
		Таким образом вычисление значения в точке для полинома Лагранжа приведенного к канонической форме будет занимать $O(n)$ операций. 
		Вычисление знчения в $N^2$ точках будет происходить за $N^2 \cdot O(n)$ итераций.
		\item Рассмотрим \textit{кубический сплайн}
		$$ S\left(x_{i}\right)=y_{i}, \quad i=0,1, \ldots, n $$
		$$ \begin{array}{c}
		s_{i}=a_{i}+b_{i}\left(x-x_{i-1}\right)+c_{i}\left(x-x_{i-1}\right)^{2}+d_{i}\left(x-x_{i-1}\right)^{3} \\
		x \in\left[x_{i-1}, x_{i}\right], \quad i=1,2, \ldots, n
		\end{array} $$ 
		
		Коэффициенты --- $a_{i}, b_{i}, c_{i}, d_{i}$ найдены на этапе интерполяции.
		
		Таким образом, вычисление $s_i(x)$ требует $O(1)$ итераций.
		$i$ так же вычисляется за $O(1)$. 
		
		Вычисления можно ускорить способом похожим на то, что предложено для \textit{полинома Лагранжа}. 
		
		Известно, что $x^{i+1} = x^i \cdot i$. Что позволяет последовательно вычислять слагаемые многочлена и ускоряет вычисления.
		
		Если до модификации вычисление $s_i$ требует $9$ операций умножения, то после требуется $6$, то есть теоретически ускоряется на $\sim\frac{1}{3}$.
		\end{enumerate}
	\item Функции $\{\phi_k(x)\}_{k=0}^{n}$ должны представлять из себя набор базисных многочленов $\phi_k(x)$, обладающих свойством, что всякий многочлен $P_n(x)$ степени $n$ может быть однозначно представлен в виде 
	$$ P_{n}(x)=\sum_{k=0}^{n} a_{k} \varphi_{k}(x) $$
	
	В качестве таких базисов можно использовать:
	\begin{enumerate}
		\item Степенной базис $\left\{x^{k}\right\}_{k=0}^{n}$
		\item Локальный степенной базис $\left\{\left(\frac{x-a}{b-a}\right)^{k}\right\}_{k=0}^{n}$ 
		\item Чебышевский базис
		$ \left\{T_{k}\left(\frac{x-(a+b) / 2}{(b-a) / 2}\right)\right\}_{k=0}^{n} $
		\item Лагранжев базис $\left\{l_{n k}(x)\right\}_{k=0}^{n}$ 
	\end{enumerate}

	Также можно построить свои базисные функции. Для этого они должны быть линейно независимы и $\forall P_n(x)$ должен являться их линейной комбинацией.
	\end{enumerate}
\end{document}