\documentclass[12pt]{article}
% Эта строка — комментарий, она не будет показана в выходном файле
\usepackage{ucs}
\usepackage[utf8x]{inputenc} % Включаем поддержку UTF8
\usepackage[russian]{babel}  % Включаем пакет для поддержки русского языка
\usepackage{amsmath, amsthm, amssymb, amsfonts}
\usepackage{multirow}
\usepackage{pbox}
\usepackage[showframe=false]{geometry}
\usepackage{changepage}
\usepackage{graphicx}
\usepackage{float}

\date{}
\author{}

\renewcommand{\thesubsection}{\arabic{subsection}}
\makeatletter
\def\@seccntformat#1{\@ifundefined{#1@cntformat}%
	{\csname the#1\endcsname\quad}%    default
	{\csname #1@cntformat\endcsname}}% enable individual control
\newcommand\section@cntformat{}     % section level 
\makeatother

\newcommand{\lagrange}[3] {
	На Рис: \ref{fig:#1chebyshev#2lagrange},  \ref{fig:#1chebyshev#3lagrange}, \ref{fig:#1uniform#2lagrange} и \ref{fig:#1uniform#3lagrange}
	представлены графики построенных полиномов Лагранжа.
	
	\begin{figure}[H]
		\centering
		\includegraphics[width=0.7\linewidth]{../results/#1_chebyshev_#2_lagrange}
		\caption{Полином Лагранжа, чебышевская сетка, $n = #2$}
		\label{fig:#1chebyshev#2lagrange}
	\end{figure}
	\begin{figure}[H]
		\centering
		\includegraphics[width=0.7\linewidth]{../results/#1_chebyshev_#3_lagrange}
		\caption{Полином Лагранжа, чебышевская сетка, $n = #3$}
		\label{fig:#1chebyshev#3lagrange}
	\end{figure}
	\begin{figure}[H]
		\centering
		\includegraphics[width=0.7\linewidth]{../results/#1_uniform_#2_lagrange}
		\caption{Полином Лагранжа, равномерная сетка, $n = #2$}
		\label{fig:#1uniform#2lagrange}
	\end{figure}
	\begin{figure}[H]
		\centering
		\includegraphics[width=0.7\linewidth]{../results/#1_uniform_#3_lagrange}
		\caption{Полином Лагранжа, равномерная сетка, $n = #3$}
		\label{fig:#1uniform#3lagrange}
	\end{figure}
}

\newcommand{\spline}[3] {
	На Рис:  \ref{fig:#1uniform#2spline} и \ref{fig:#1uniform#3spline} представлены результаты Сплайн интерполяции.
	
	\begin{figure}[H]
		\centering
		\includegraphics[width=0.7\linewidth]{../results/#1_uniform_#2_spline}
		\caption{Сплйн интерполяция, равномерная сетка, $n = #2$}
		\label{fig:#1uniform#2spline}
	\end{figure}
	\begin{figure}[H]
		\centering
		\includegraphics[width=0.7\linewidth]{../results/#1_uniform_#3_spline}
		\caption{Сплайн интерполяция, равномерная сетка, $n = #3$}
		\label{fig:#1uniform#3spline}
	\end{figure}
}

\begin{document}
\section{Результаты выполнения лабораторной работы №4}
\subsection{Интерполяция $f(x) = 1$ и $f(x) = x$ на $[-1, 1]$ многочленом Лагранжа}

\subsubsection{Исследование функции $f(x) = 1$ на $[-1, 1]$}
Результаты:\\
\begin{tabular}{|l|l|l|}
\hline Количество узлов n & \pbox{20cm}{Норма ошибки\\ на равномерной сетке} & \pbox{20cm}{Норма ошибки\\ на чебышевской сетке} \\ \hline
4 & 2.22045e-16 & 6.66134e-16 \\ \hline
8 & 1.33227e-15 & 8.88178e-16 \\ \hline
16 & 1.66978e-13 & 1.77636e-15 \\ \hline
32 & 4.59099e-09 & 1.9984e-15 \\ \hline
64 & 9.38634 & 2.88658e-15 \\ \hline
128 & 6.58355e+19 & 5.32907e-15 \\ \hline
\end{tabular}\\
\lagrange{const}{4}{64}

\subsubsection{Исследование функции $f(x) = x$ на $[-1, 1]$}
Результаты:\\
\begin{tabular}{|l|l|l|}
\hline Количество узлов n & \pbox{20cm}{Норма ошибки\\ на равномерной сетке} & \pbox{20cm}{Норма ошибки\\ на чебышевской сетке} \\ \hline
4 & 2.22045e-16 & 5.55112e-16 \\ \hline
8 & 6.66134e-16 & 7.77156e-16 \\ \hline
16 & 2.9976e-14 & 1.11022e-15 \\ \hline
32 & 5.10888e-10 & 1.66533e-15 \\ \hline
64 & 0.957783 & 2.44249e-15 \\ \hline
128 & 3.53617e+18 & 3.33067e-15 \\ \hline
\end{tabular}\\

\lagrange{linear}{4}{64}

\subsection{Исследование функции Рунге $f(x) = \frac{1}{1 + 25x^2}$}
\lagrange{runge}{4}{32}
\spline{runge}{4}{32}



\subsection{Сплайн интерполяция $f(x) = x$ и $f(x) = x^2$ на $[-1, 1]$}

\subsubsection{$f(x) = x$ на $[-1, 1]$}
\begin{tabular}{|l|l|l|l|l|}
	\hline 
	Итер.(n) & Шаг сетки $h_n$ & Норма ошибки $err_n$ &  Отношение ошибок $z_n$ & Порядок сход. $p_n$ \\ \hline
	1 (4)  & 0.5 & 0 & 0 & 0 \\ \hline
	2 (8)  & 0.25 & 0 & 0 & 0 \\ \hline
	3 (16)  & 0.125 & 0 & 0 & 0 \\ \hline
	4 (32)  & 0.0625 & 0 & 0 & 0 \\ \hline
	5 (64)  & 0.03125 & 0 & 0 & 0 \\ \hline
	6 (128)  & 0.015625 & 0 & 0 & 0 \\ \hline
\end{tabular}\\

\spline{linear}{4}{32}

\subsubsection{$f(x) = x^2$ на $[-1, 1]$}
\begin{tabular}{|l|l|l|l|l|}
	\hline 
	Итер.(n) & Шаг сетки $h_n$ & Норма ошибки $err_n$ &  Отношение ошибок $z_n$ & Порядок сход. $p_n$ \\ \hline
	1 (4)  & 0.5 & 0.0240653 & 0.0240653 & 5.3769 \\ \hline
	2 (8)  & 0.25 & 0.00613498 & 0.254931 & 1.97182 \\ \hline
	3 (16)  & 0.125 & 0.0015339 & 0.250025 & 1.99985 \\ \hline
	4 (32)  & 0.0625 & 0.000383475 & 0.25 & 2 \\ \hline
	5 (64)  & 0.03125 & 9.58687e-05 & 0.25 & 2 \\ \hline
	6 (128)  & 0.015625 & 2.23404e-05 & 0.233031 & 2.1014 \\ \hline
\end{tabular}\\

\spline{quad}{4}{32}

\subsection{Интерполяция $f(x) = \sin{\pi x}$ на $[-1, 1]$ и $[-1.25, 1.25]$}
\subsubsection{Сплайн интерполяция $f(x) = \sin{\pi x}$ на $[-1, 1]$}
\begin{tabular}{|l|l|l|l|l|}
	\hline 
	Итер.(n) & Шаг сетки $h_n$ & Норма ошибки $err_n$ &  Отношение ошибок $z_n$ & Порядок сход. $p_n$ \\ \hline
	1 (4)  & 0.5 & 0.0200162 & 0.0200162 & 5.64268 \\ \hline
	2 (8)  & 0.25 & 0.00106609 & 0.053261 & 4.23078 \\ \hline
	3 (16)  & 0.125 & 6.31129e-05 & 0.0592006 & 4.07824 \\ \hline
	4 (32)  & 0.0625 & 3.8893e-06 & 0.0616244 & 4.02035 \\ \hline
	5 (64)  & 0.03125 & 2.42209e-07 & 0.0622759 & 4.00518 \\ \hline
	6 (128)  & 0.015625 & 1.51244e-08 & 0.0624436 & 4.0013 \\ \hline
\end{tabular}\\

\spline{sin1}{4}{32}

\subsubsection{Сплайн интерполяция $f(x) = \sin{\pi x}$ на $[-1.25, 1.25]$}
\begin{tabular}{|l|l|l|l|l|}
	\hline 
	Итер.(n) & Шаг сетки $h_n$ & Норма ошибки $err_n$ &  Отношение ошибок $z_n$ & Порядок сход. $p_n$ \\ \hline
	1 (4)  & 0.625 & 0.159895 & 0.159895 & 2.6448 \\ \hline
	2 (8)  & 0.3125 & 0.0368117 & 0.230224 & 2.11889 \\ \hline
	3 (16)  & 0.15625 & 0.00859868 & 0.233586 & 2.09798 \\ \hline
	4 (32)  & 0.078125 & 0.00210632 & 0.244958 & 2.02939 \\ \hline
	5 (64)  & 0.0390625 & 0.000523695 & 0.248631 & 2.00792 \\ \hline
	6 (128)  & 0.0195312 & 0.000121869 & 0.23271 & 2.1034 \\ \hline
\end{tabular}\\

\spline{sin125}{4}{32}

\subsubsection{Интерполяция  $f(x) = \sin{\pi x}$ полиномом Лагранжа на $[-1, 1]$}
\textit{Равномерная сетка:}\\


\begin{tabular}{|l|l|l|l|l|}
	\hline 
	Итер.(n) & Шаг сетки $h_n$ & Норма ошибки $err_n$ &  Отношение ошибок $z_n$ & Порядок сход. $p_n$ \\ \hline
	1 (4)  & 0.5 & 0.180758 & 0.180758 & 2.46787 \\ \hline
	2 (8)  & 0.25 & 0.00120515 & 0.00666722 & 7.2287 \\ \hline
	3 (16)  & 0.125 & 6.65075e-10 & 5.5186e-07 & 20.7892 \\ \hline
	4 (32)  & 0.0625 & 1.46266e-09 & 2.19925 & -1.13701 \\ \hline
	5 (64)  & 0.03125 & 2.66712 & 1.82346e+09 & -30.764 \\ \hline
	6 (128)  & 0.015625 & 1.11681e+19 & 4.18733e+18 & -61.8607 \\ \hline
\end{tabular}\\


\textit{Чебышевская сетка:}\\


\begin{tabular}{|l|l|l|}
	\hline 
	Итер.(n) & Норма ошибки $err_n$ &  Отношение ошибок $z_n$ \\ \hline
	1 (4)  &  0.115553 & 0.115553 \\ \hline
	2 (8)  &  0.00026113 & 0.00225982  \\ \hline
	3 (16)  & 1.07235e-11 & 4.10659e-08  \\ \hline
	4 (32)  &  1.88738e-15 & 0.000176003  \\ \hline
	5 (64)  &  2.10942e-15 & 1.11765 \\ \hline
	6 (128)  & 3.21965e-15 & 1.52632  \\ \hline
\end{tabular}\\

\lagrange{sin1}{4}{32}

\subsection{Задача варианта}
$f(x) = \sin(\frac{2 * x^2 - x + 2 \cdot 7^{\frac{1}{3}} - 5}{2}) + e^{\frac{x^2 + 2 \cdot x + 1}{7 \cdot x + 1}} - 1.5$

Функция разрывна в $x = -1/7$. Построенные алгоритмы интерполяции не позволяют без дополнительных модификаций найти достаточно близкое приближение функции.

\spline{target}{4}{32}
\lagrange{target}{4}{32}


\end{document}