\documentclass[12pt]{article}
% Эта строка — комментарий, она не будет показана в выходном файле
\usepackage{ucs}
\usepackage[utf8x]{inputenc} % Включаем поддержку UTF8
\usepackage[russian]{babel}  % Включаем пакет для поддержки русского языка
\usepackage{amsmath, amsthm, amssymb, amsfonts}
\usepackage{multirow}
\usepackage{pbox}
\usepackage[showframe=false]{geometry}
\usepackage{changepage}
\usepackage{graphicx}

\date{}
\author{}

\renewcommand{\thesubsection}{\arabic{subsection}}
\makeatletter
\def\@seccntformat#1{\@ifundefined{#1@cntformat}%
	{\csname the#1\endcsname\quad}%    default
	{\csname #1@cntformat\endcsname}}% enable individual control
\newcommand\section@cntformat{}     % section level 
\makeatother


\begin{document}
\subsection{Результаты выполнения лабораторной работы №4}

\subsubsection{Исследование функции $f(x) = 1$ на $[-1, 1]$}
Результаты:\\
\begin{tabular}{|l|l|l|}
\hline Количество узлов n & \pbox{20cm}{Норма ошибки\\ на равномерной сетке} & \pbox{20cm}{Норма ошибки\\ на чебышевской сетке} \\ \hline
4 & 0.438332 & 0.402016 \\ \hline
8 & 1.04486 & 0.17083 \\ \hline
16 & 14.3699 & 0.0326112 \\ \hline
32 & 5026.8 & 0.00140123 \\ \hline
64 & 1.04155e+09 & 2.45011e-06 \\ \hline
128 & 9.93261e+19 & 7.24065e-12 \\ \hline
\end{tabular}\\


На Рис: \ref{fig:constchebyshev4lagrange},  \ref{fig:constchebyshev128lagrange}, \ref{fig:constuniform4lagrange} и \ref{fig:constuniform128lagrange}
представлены графики построенных полиномов.

\begin{figure}
	\centering
	\includegraphics[width=0.7\linewidth]{../results/const_chebyshev_4_lagrange}
	\caption{Полином Лагранжа, чебышевская сетка, $n = 4$}
	\label{fig:constchebyshev4lagrange}
\end{figure}
\begin{figure}
	\centering
	\includegraphics[width=0.7\linewidth]{../results/const_chebyshev_128_lagrange}
	\caption{Полином Лагранжа, чебышевская сетка, $n = 128$}
	\label{fig:constchebyshev128lagrange}
\end{figure}
\begin{figure}
	\centering
	\includegraphics[width=0.7\linewidth]{../results/const_uniform_4_lagrange}
	\caption{Полином Лагранжа, равномерная сетка, $n = 4$}
	\label{fig:constuniform4lagrange}
\end{figure}
\begin{figure}
	\centering
	\includegraphics[width=0.7\linewidth]{../results/const_uniform_128_lagrange}
	\caption{Полином Лагранжа, равномерная сетка, $n = 128$}
	\label{fig:constuniform128lagrange}
\end{figure}


\end{document}